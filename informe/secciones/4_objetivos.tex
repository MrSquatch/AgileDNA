\section{Objetivos}

\subsection*{Objetivo General}
Diseñar e implementar un sistema de asignación automática y óptima de tareas en entornos ágiles tipo SCRUM, utilizando algoritmos genéticos que consideren múltiples restricciones reales como carga de trabajo, habilidades requeridas, prioridades, dependencias y costos.

\subsection*{Objetivos Específicos}
\begin{itemize}
    \item Balancear la carga de trabajo entre los desarrolladores, considerando sus capacidades horarias individuales.
    \item Asignar tareas a los desarrolladores cuyas habilidades coincidan en mayor medida con los requerimientos técnicos de cada tarea.
    \item Minimizar el tiempo total de ejecución del conjunto de tareas.
    \item Garantizar que las dependencias entre tareas se respeten en la planificación del sprint.
    \item Minimizar el costo total del proyecto considerando el costo por hora de cada desarrollador.
    \item Diseñar una función de evaluación que combine múltiples criterios mediante un sistema de pesos ajustables.
\end{itemize}


% \section{Objetivos}

% \subsection*{Objetivo General}
% Optimizar la asignación de tareas en entornos SCRUM utilizando algoritmos genéticos.

% \subsection*{Objetivos Específicos}
% \begin{itemize}
%     \item Balancear la carga de trabajo entre programadores.
%     \item Minimizar el tiempo total del sprint.
%     \item Considerar dependencias entre tareas en la asignación.
% \end{itemize}
% \vspace{0.5cm}

% \begin{tcolorbox}[colback=gray!10, colframe=black!30, title={Sugerencia para esta sección}]
%     El objetivo general debe resumir lo que se quiere lograr. Los específicos deben ser medibles y relacionados con el algoritmo y el contexto ágil.
% \end{tcolorbox}
