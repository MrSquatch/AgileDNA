\section{Definición del Problema}
En entornos ágiles como SCRUM, una de las dificultades más recurrentes es la asignación eficiente y equitativa de tareas entre los miembros del equipo de desarrollo. Esta asignación debe considerar múltiples factores simultáneamente, tales como la capacidad individual de cada desarrollador, la urgencia de las tareas, las dependencias entre actividades, las habilidades requeridas y los costos asociados al trabajo.

En la práctica, esta distribución suele realizarse de forma manual o semi-automática, lo que da lugar a diversos problemas: algunos desarrolladores terminan sobrecargados mientras otros están subutilizados, las tareas críticas o urgentes pueden quedar rezagadas, y las dependencias mal gestionadas provocan cuellos de botella en el flujo de trabajo del sprint. Además, si se ignora el nivel de habilidad requerido para cada tarea, pueden generarse asignaciones ineficientes que comprometen la calidad del producto o retrasan su entrega.

Este problema se vuelve aún más complejo al considerar que cada tarea puede requerir habilidades específicas en diferentes niveles, que la capacidad de los desarrolladores no siempre es uniforme (por ejemplo, algunos trabajan 30 horas por sprint, otros 40), y que el costo por hora de cada colaborador puede influir significativamente en el presupuesto del proyecto.

Frente a esta complejidad combinatoria y multifactorial, se requiere un enfoque de optimización inteligente que permita encontrar asignaciones que equilibren el uso de recursos, minimicen el tiempo total del proyecto, atiendan prioridades, y reduzcan costos. En este contexto, el uso de algoritmos genéticos se presenta como una alternativa viable para explorar eficientemente un espacio de soluciones altamente complejo.

% En entornos ágiles como SCRUM, una de las dificultades recurrentes es la asignación eficiente de tareas entre los miembros del equipo, considerando factores como carga de trabajo, tiempo estimado, prioridades y dependencias.
% \vspace{0.5cm}

% \begin{tcolorbox}[colback=gray!10, colframe=black!30, title={Sugerencia para esta sección}]
%     Describe el problema real: cómo se distribuyen mal las tareas, cómo puede haber sobrecarga en algunos programadores y cómo las dependencias entre tareas generan cuellos de botella.
% \end{tcolorbox}
