\section{Metodología}

La solución propuesta se desarrolla en etapas que siguen una secuencia lógica de análisis, diseño, implementación y validación. Inicialmente, se definen los parámetros de entrada: tareas, programadores, tiempos estimados y dependencias. Luego, se formula el problema como una función objetivo y se implementa un algoritmo genético con operadores personalizados.

Se incluye un flujograma general del sistema, así como una arquitectura conceptual que modela la interacción entre módulos. Finalmente, se valida la solución a través de casos simulados.

\vspace{0.5cm}

\begin{tcolorbox}[colback=gray!10, colframe=black!30, title={Sugerencias para esta sección}]
    \begin{itemize}
        \item Usa un flujograma para mostrar el proceso general (puedes incluirlo más adelante).
        \item Describe cómo se estructura el algoritmo: población, selección, cruza, mutación, etc.
        \item Explica cómo se simula el entorno SCRUM para validar resultados.
        \item Detalla cualquier herramienta o lenguaje usado (Python, etc.).
    \end{itemize}
\end{tcolorbox}