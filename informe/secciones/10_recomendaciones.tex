\section{Recomendaciones}

Una de las principales recomendaciones es extender la solución para permitir la carga dinámica de datos por parte del usuario. Actualmente, las tareas y desarrolladores están predefinidos en el sistema, lo cual limita su flexibilidad. Incorporar formularios o integración con bases de datos permitiría simular escenarios reales y adaptar el sistema a contextos específicos de empresas o instituciones educativas. Esta mejora también facilitaría la validación del algoritmo con conjuntos de datos reales y facilitaría su adopción práctica.

Adicionalmente, se sugiere mejorar las capacidades de visualización del sistema. Aunque la interfaz actual permite observar resultados básicos, sería conveniente integrar herramientas interactivas como gráficos de Gantt, dashboards con métricas de rendimiento o incluso simulaciones visuales del avance del sprint. Estas mejoras no solo enriquecerían la experiencia del usuario, sino que también permitirían tomar decisiones más informadas a partir de los resultados generados por el algoritmo.

Finalmente, desde una perspectiva investigativa, se recomienda experimentar con variantes del algoritmo genético e incluso explorar enfoques híbridos, como algoritmos meméticos o combinaciones con reglas heurísticas extraídas de prácticas ágiles reales. También sería interesante aplicar el sistema en casos de estudio reales, con equipos de desarrollo reales, para evaluar no solo métricas técnicas como el makespan o la varianza de carga, sino también el impacto en la motivación, percepción de equidad y productividad del equipo.

% \section{Recomendaciones}

% A partir del análisis y la implementación realizada, se proponen las siguientes recomendaciones para futuras investigaciones o mejoras en sistemas SCRUM asistidos por algoritmos genéticos:

% \begin{itemize}
%     \item Integrar herramientas de visualización para facilitar la interpretación de las asignaciones realizadas por el algoritmo.
%     \item Explorar técnicas híbridas que combinen algoritmos genéticos con heurísticas basadas en reglas de negocio.
%     \item Validar el modelo con datos reales de proyectos ágiles en diferentes industrias.
%     \item Evaluar el impacto del algoritmo en la satisfacción y productividad del equipo.
% \end{itemize}

% \vspace{0.5cm}

% \begin{tcolorbox}[colback=gray!10, colframe=black!30, title={Sugerencia para esta sección}]
%     Formula recomendaciones específicas, factibles y que se deriven directamente de los resultados de tu propuesta. Considera aspectos como mejoras técnicas, escalabilidad o nuevas direcciones de investigación.
% \end{tcolorbox}
