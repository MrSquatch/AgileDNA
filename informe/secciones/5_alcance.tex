\section{Alcance del Proyecto}

Este trabajo aborda la implementación de un sistema de asignación automática de tareas en entornos SCRUM, considerando aspectos clave como capacidades horarias variables entre desarrolladores, habilidades técnicas diferenciadas, esfuerzo estimado de las tareas en horas, complejidad y prioridades. El sistema emplea un algoritmo genético como núcleo de optimización, buscando asignaciones que equilibren la carga, respeten dependencias y minimicen el costo total del proyecto.

El desarrollo se limita a la simulación de un equipo de desarrollo y no contempla integración directa con plataformas ágiles comerciales ni modelado de otros roles más allá de los desarrolladores. Tampoco se abordan requerimientos no funcionales como persistencia de datos, seguridad o despliegue en producción, ya que el enfoque es validar la viabilidad técnica del enfoque propuesto como un \textit{proof of concept} adaptable a escenarios reales.


% \section{Alcance del Proyecto}
% Este proyecto se enfocará en la asignación de tareas para un equipo ágil ficticio conformado por desarrolladores, analistas y testers. No se abordarán aspectos de gestión externa o requerimientos no funcionales.
% \vspace{0.5cm}

% \begin{tcolorbox}[colback=gray!10, colframe=black!30, title={Sugerencia para esta sección}]
%     Define claramente qué incluye tu solución (asignación interna de tareas, esfuerzo estimado, simulación SCRUM) y qué no (por ejemplo, herramientas reales, despliegue).
% \end{tcolorbox}
