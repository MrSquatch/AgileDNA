\newpage
\section*{Resumen}

Este trabajo presenta una propuesta de sistema inteligente para la asignación de tareas en equipos ágiles, utilizando un algoritmo genético multiobjetivo como núcleo de optimización. A través de una arquitectura cliente-servidor compuesta por un frontend en React y una API en Flask, el usuario puede configurar parámetros clave del algoritmo y visualizar los resultados de forma interactiva. La solución considera restricciones como habilidades requeridas, dependencias entre tareas, balance de carga y costos. Las pruebas realizadas con datos simulados permitieron validar el comportamiento del sistema ante distintas configuraciones, demostrando su utilidad como prueba de concepto aplicable a entornos reales de desarrollo ágil.

\vspace{0.5cm}

\textbf{Palabras clave:} \textit{algoritmos genéticos, asignación de tareas, SCRUM, optimización, metodología ágil}

\vspace{1cm}

\section*{Abstract}

This work presents an intelligent task assignment system for agile teams, based on a multi-objective genetic algorithm as the optimization core. Through a client-server architecture composed of a React frontend and a Flask API, users can configure key algorithm parameters and visualize the results interactively. The solution takes into account constraints such as required skills, task dependencies, workload balance, and cost. Experiments with simulated data validated the system's behavior under different configurations, demonstrating its potential as a proof of concept for real-world agile development environments.

\vspace{0.5cm}

\textbf{Keywords:} \textit{genetic algorithms, task allocation, SCRUM, optimization, agile methodology}

\newpage


% \newpage
% \section*{Resumen}

% En el presente trabajo se explora el uso de algoritmos genéticos para resolver el problema de asignación óptima de tareas dentro de equipos ágiles. La metodología empleada permite balancear la carga entre desarrolladores, reducir tiempos de entrega y mejorar la gestión de dependencias entre tareas. Los resultados preliminares muestran una mejora significativa frente a enfoques tradicionales.

% \vspace{0.5cm}

% \textbf{Palabras clave:} \textit{algoritmos genéticos, asignación de tareas, SCRUM, metodologías ágiles}

% \vspace{1cm}

% \section*{Abstract}

% This paper explores the use of genetic algorithms to solve the problem of optimal task assignment within agile teams. The methodology allows for workload balancing, reduction of delivery times, and improved management of task dependencies. Preliminary results show a significant improvement compared to traditional approaches.

% \vspace{0.5cm}

% \textbf{Keywords:} \textit{genetic algorithms, task allocation, SCRUM, agile methodologies}

% \vspace{1cm}

% \begin{tcolorbox}[colback=gray!10, colframe=black!30, title={Sugerencias para esta sección}]
%     \begin{itemize}
%         \item Reemplaza los textos por un resumen real cuando finalices el informe.
%         \item En español e inglés, resume: objetivo, metodología, resultados clave y conclusiones.
%         \item Evita listas o fórmulas; usa prosa clara, breve y técnica.
%         \item Las palabras clave deben estar relacionadas con los principales temas del proyecto.
%     \end{itemize}
% \end{tcolorbox}

% \newpage
