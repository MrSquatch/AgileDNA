\section{Conclusiones}

Este trabajo ha demostrado que es posible construir un sistema automatizado capaz de resolver la asignación de tareas en entornos ágiles SCRUM mediante un algoritmo genético multiobjetivo, integrando simultáneamente múltiples restricciones prácticas como habilidades requeridas, dependencias entre tareas, capacidades individuales y costos. La solución no solo logra generar asignaciones eficientes, sino que también respeta principios clave de metodologías ágiles como la iteración progresiva y la transparencia del proceso.

La implementación completa, compuesta por una interfaz gráfica en React, una API desarrollada en Flask y un motor de optimización en Python, valida que un enfoque modular y desacoplado es viable incluso en proyectos académicos, sentando las bases para futuras integraciones más complejas. Este diseño permitió realizar validaciones constantes a lo largo del desarrollo, ajustando y refinando componentes del algoritmo hasta alcanzar una versión estable y funcional.

Uno de los principales aprendizajes fue comprobar que, con una buena representación del problema y operadores genéticos bien diseñados, es posible obtener soluciones de alta calidad incluso en un dominio con restricciones duras y objetivos conflictivos. También se confirmó que la exposición clara de los parámetros de configuración al usuario (como tasas de cruce, mutación o pesos de la función objetivo) agrega valor al sistema, haciéndolo más flexible y comprensible.

Si bien existen limitaciones como el uso de datos estáticos para tareas y desarrolladores, estas no comprometen el valor de la propuesta como prueba de concepto sólida. El sistema desarrollado no solo cumple su objetivo académico, sino que representa un punto de partida válido para aplicaciones reales en entornos empresariales o educativos donde se requiere automatizar la gestión de tareas de forma inteligente.

% \section{Conclusiones}

% El enfoque basado en algoritmos genéticos ha demostrado ser una herramienta eficiente para la asignación de tareas en entornos SCRUM. Se ha logrado un balance adecuado en la distribución del trabajo, respetando restricciones y optimizando el tiempo total del proyecto.

% Este trabajo sienta las bases para aplicaciones reales en la industria del software, y abre la posibilidad de integrar técnicas similares en procesos más amplios de gestión ágil.

% \vspace{0.5cm}

% \begin{tcolorbox}[colback=gray!10, colframe=black!30, title={Sugerencias para esta sección}]
%     \begin{itemize}
%         \item Resume los aprendizajes clave del trabajo realizado.
%         \item Redacta al menos 3 conclusiones claras y bien separadas si es posible.
%         \item Puedes mencionar limitaciones y oportunidades de mejora.
%         \item Evita repetir el resumen o los objetivos textualmente.
%     \end{itemize}
% \end{tcolorbox}