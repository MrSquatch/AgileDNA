\newpage

\begingroup
% Espacio extra entre referencias (pero sin interlineado doble)
\setlength{\bibitemsep}{8pt}   % ≈ medio renglón; cambia 8 pt a lo que prefieras
\section*{Bibliografía}
\printbibliography[heading=none]
\endgroup

\vspace{0.5cm}

\begin{tcolorbox}[colback=gray!10, colframe=black!30, title={Sugerencia para esta sección}]
    Usa el estilo APA 7 para citar fuentes académicas. Puedes utilizar herramientas como Zotero, Mendeley o Google Scholar para exportar referencias en formato BibTeX (.bib) y guardarlas en el archivo \texttt{bibliografia.bib}, ubicado en la carpeta \texttt{referencias}.
\end{tcolorbox}

\newpage
