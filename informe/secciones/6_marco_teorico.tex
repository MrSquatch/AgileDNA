\section{Marco Teórico}

Según \textcite{scrum2020}, la asignación de tareas dentro de metodologías ágiles como SCRUM requiere un entendimiento profundo de conceptos relacionados con planificación de proyectos, inteligencia artificial y optimización. Los algoritmos genéticos, inspirados en los procesos evolutivos naturales, ofrecen una alternativa poderosa para resolver problemas complejos con múltiples variables y restricciones \parencite{goldberg}.

% … o, si quieres el apellido dentro del texto:
% Según \textcite{goldberg}, los algoritmos genéticos ofrecen …

Dentro del marco de este trabajo, se exploran teorías relacionadas con el problema del viajante (TSP), la gestión ágil de proyectos, técnicas de optimización y fundamentos de los algoritmos genéticos aplicados a software.

\vspace{0.5cm}

\begin{tcolorbox}[colback=gray!10,colframe=black!30,title={Sugerencias para esta sección}]
    \begin{itemize}
        \item Introduce brevemente conceptos como SCRUM, DevOps, optimización, TSP, etc.
        \item Cita autores relevantes con formato APA 7 si es necesario.
        \item Relaciona estos conceptos con el problema que estás abordando.
        \item Puedes usar gráficos teóricos o tablas si fuera necesario.
    \end{itemize}
\end{tcolorbox}
