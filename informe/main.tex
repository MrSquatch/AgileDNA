\documentclass[12pt]{article}
\usepackage{styles/plantilla-profesor}

\usepackage{graphicx}
\usepackage[spanish]{babel}

\begin{document}

\begin{titlepage}
    \centering
    {\LARGE Universidad Nacional Mayor de San Marcos \par}
    {\large Universidad del Perú. Decana de América\par}
    \vspace{0.5cm}
    {\Large Facultad de Ingeniería de Sistemas e Informática\par}
    \vspace{0.5cm}

    \includegraphics[width=8cm]{imagenes/logo-unmsm.png}\par
    \vspace{0.5cm}

    {\LARGE\bfseries Asignación Óptima de Recursos y Tareas en Metodologías Ágiles\par}
    \vspace{0.5cm}
    {\large\bfseries Aplicación de Algoritmos Genéticos en Entornos SCRUM\par}
    \vspace{1cm}

    {\Large\bfseries Integrantes:\par}
    \vspace{0.5cm}
    Salazar Garcia, Diego\\[4pt]
    Arias Chumpitaz, Giovanni\\[4pt]
    Arroyo Vasquez, Luis\\[4pt]
    Lavaud Guevara, Jean\\[4pt]
    \vspace{1cm}

    {\large \textbf{Docente:} Claudio Arango\par}
    \vfill

    {\large Lima, Perú\par}
    {\large 2025\par}
\end{titlepage}


\section*{Resumen}
Este informe presenta una solución basada en algoritmos genéticos para abordar el problema de congestión vehicular en Lima. Se describe el contexto del tráfico urbano y la aplicación de técnicas de optimización bioinspiradas.


\section{Definición del Problema}
En entornos ágiles como SCRUM, una de las dificultades más recurrentes es la asignación eficiente y equitativa de tareas entre los miembros del equipo de desarrollo. Esta asignación debe considerar múltiples factores simultáneamente, tales como la capacidad individual de cada desarrollador, la urgencia de las tareas, las dependencias entre actividades, las habilidades requeridas y los costos asociados al trabajo.

En la práctica, esta distribución suele realizarse de forma manual o semi-automática, lo que da lugar a diversos problemas: algunos desarrolladores terminan sobrecargados mientras otros están subutilizados, las tareas críticas o urgentes pueden quedar rezagadas, y las dependencias mal gestionadas provocan cuellos de botella en el flujo de trabajo del sprint. Además, si se ignora el nivel de habilidad requerido para cada tarea, pueden generarse asignaciones ineficientes que comprometen la calidad del producto o retrasan su entrega.

Este problema se vuelve aún más complejo al considerar que cada tarea puede requerir habilidades específicas en diferentes niveles, que la capacidad de los desarrolladores no siempre es uniforme (por ejemplo, algunos trabajan 30 horas por sprint, otros 40), y que el costo por hora de cada colaborador puede influir significativamente en el presupuesto del proyecto.

Frente a esta complejidad combinatoria y multifactorial, se requiere un enfoque de optimización inteligente que permita encontrar asignaciones que equilibren el uso de recursos, minimicen el tiempo total del proyecto, atiendan prioridades, y reduzcan costos. En este contexto, el uso de algoritmos genéticos se presenta como una alternativa viable para explorar eficientemente un espacio de soluciones altamente complejo.

% En entornos ágiles como SCRUM, una de las dificultades recurrentes es la asignación eficiente de tareas entre los miembros del equipo, considerando factores como carga de trabajo, tiempo estimado, prioridades y dependencias.
% \vspace{0.5cm}

% \begin{tcolorbox}[colback=gray!10, colframe=black!30, title={Sugerencia para esta sección}]
%     Describe el problema real: cómo se distribuyen mal las tareas, cómo puede haber sobrecarga en algunos programadores y cómo las dependencias entre tareas generan cuellos de botella.
% \end{tcolorbox}

\section*{Objetivos}
\begin{itemize}
  \item Desarrollar un algoritmo genético para optimizar rutas.
  \item Simular su comportamiento bajo distintas condiciones urbanas.
  \item Validar los resultados con datos reales de tráfico.
\end{itemize}

\section{Alcance del Proyecto}

Este trabajo aborda la implementación de un sistema de asignación automática de tareas en entornos SCRUM, considerando aspectos clave como capacidades horarias variables entre desarrolladores, habilidades técnicas diferenciadas, esfuerzo estimado de las tareas en horas, complejidad y prioridades. El sistema emplea un algoritmo genético como núcleo de optimización, buscando asignaciones que equilibren la carga, respeten dependencias y minimicen el costo total del proyecto.

El desarrollo se limita a la simulación de un equipo de desarrollo y no contempla integración directa con plataformas ágiles comerciales ni modelado de otros roles más allá de los desarrolladores. Tampoco se abordan requerimientos no funcionales como persistencia de datos, seguridad o despliegue en producción, ya que el enfoque es validar la viabilidad técnica del enfoque propuesto como un \textit{proof of concept} adaptable a escenarios reales.


% \section{Alcance del Proyecto}
% Este proyecto se enfocará en la asignación de tareas para un equipo ágil ficticio conformado por desarrolladores, analistas y testers. No se abordarán aspectos de gestión externa o requerimientos no funcionales.
% \vspace{0.5cm}

% \begin{tcolorbox}[colback=gray!10, colframe=black!30, title={Sugerencia para esta sección}]
%     Define claramente qué incluye tu solución (asignación interna de tareas, esfuerzo estimado, simulación SCRUM) y qué no (por ejemplo, herramientas reales, despliegue).
% \end{tcolorbox}


\section*{2. Marco Teórico}
Se exploran conceptos como el Problema del Viajante (TSP), principios de DevOps para integración y despliegue continuo, y fundamentos de algoritmos genéticos. La relevancia de la automatización se respalda con autores en formato APA 7.

\section*{3. Metodología}
Se sigue un enfoque incremental: definición de objetivos, modelado del TSP, diseño del algoritmo genético y validación. Se incluye flujograma y arquitectura de la solución.

\section{Propuesta de Solución}

La propuesta presentada consiste en un sistema interactivo orientado a entornos ágiles, que permite optimizar la asignación de tareas dentro de un equipo de desarrollo mediante el uso de un algoritmo genético multiobjetivo. La solución busca balancear la carga de trabajo, minimizar el tiempo total de ejecución y considerar factores clave como dependencias, habilidades requeridas y costos asociados.

El sistema se estructura en torno a una arquitectura modular que separa claramente la interfaz de usuario, la lógica de negocio y el motor de optimización. Esta separación permite una interacción clara y eficiente con el usuario, al mismo tiempo que encapsula la complejidad de la heurística evolutiva implementada.

En las siguientes subsecciones se detalla el modelo general del sistema, el diseño del algoritmo genético propuesto, y la forma en que los componentes se integran para generar una solución viable y extensible al problema de asignación de tareas.

\subsection{Modelo General del Sistema}

Como se representó anteriormente en la Figura~\ref{fig:arquitectura}, el sistema está compuesto por tres componentes principales: \textbf{Interfaz Gráfica}, \textbf{API} y \textbf{Núcleo de Optimización}.

\begin{itemize}
    \item \textbf{Interfaz Gráfica}: Permite al usuario visualizar los resultados y configurar parámetros clave del algoritmo genético, como el tamaño de la población, número de generaciones, tasas de cruce y mutación, tamaño del torneo y los pesos de la función objetivo.

    \item \textbf{API}: Recibe las configuraciones del usuario y los datos estáticos de tareas y desarrolladores. Actúa como intermediario entre la interfaz gráfica y el núcleo del algoritmo, enviando y recibiendo información a través de endpoints definidos en Flask.

    \item \textbf{Núcleo de Optimización}: Implementado en Python, este módulo contiene la lógica completa del algoritmo genético. Se encarga de generar soluciones viables y optimizadas para la asignación de tareas, respetando restricciones de dependencias, habilidades, tiempos y costos.
\end{itemize}

Esta aproximación cliente-servidor con separación clara entre capas permite al usuario interactuar de forma sencilla con la aplicación, facilitando la configuración de parámetros y la visualización de resultados, todo mientras se mantiene encapsulada la complejidad interna del algoritmo genético multiobjetivo para la optimización de la asignación de tareas.

\subsection{Algoritmo Genético Propuesto}

El núcleo del sistema implementa un \textbf{algoritmo genético multiobjetivo} para resolver el problema de asignación de tareas bajo múltiples restricciones prácticas. Esta técnica de optimización evolutiva busca simultáneamente minimizar o balancear varios factores conflictivos mediante una \textit{función de evaluación compuesta}, lo cual distingue este enfoque de soluciones más simples o unidimensionales.

Como ya fue ilustrado en la Figura~\ref{fig:flujo}, el algoritmo parte de una población aleatoria de asignaciones. A través de ciclos generacionales, se aplican los operadores clásicos de selección, cruce y mutación, buscando mejorar progresivamente la calidad de las soluciones candidatas.

\begin{itemize}
    \item \textbf{Función de evaluación compuesta (multiobjetivo):} La aptitud de cada cromosoma se determina mediante una combinación ponderada de los siguientes objetivos:
          \begin{itemize}
              \item \textit{Makespan (tiempo total):} Representa la duración total del sprint generado. Minimizarlo busca reducir el tiempo de entrega general del proyecto.
              \item \textit{Varianza de carga:} Evalúa el equilibrio en el esfuerzo asignado a cada desarrollador. Minimizarla ayuda a evitar sobrecargas individuales.
              \item \textit{Coincidencia de habilidades:} Penaliza asignaciones en las que los desarrolladores no cumplen con los niveles de habilidad requeridos por las tareas. Esto incentiva una mejor correspondencia entre competencias y responsabilidades.
              \item \textit{Costo total estimado:} Calculado en función del tiempo trabajado y el costo por hora de cada desarrollador. Busca reducir el impacto económico del plan generado.
          \end{itemize}
          Los pesos asignados a cada uno de estos criterios pueden ser definidos por el usuario a través de la interfaz del sistema.

    \item \textbf{Verificación previa y caché:} Se implementa una caché que almacena los valores de fitness ya evaluados para evitar recomputaciones innecesarias. Asimismo, las tareas se ordenan topológicamente antes del proceso evolutivo para asegurar que las dependencias entre ellas se respeten desde el inicio.

    \item \textbf{Penalización por soluciones inválidas:} Si una asignación excede la capacidad de un desarrollador o presenta un desajuste extremo de habilidades, se le asigna una penalización severa (\texttt{BIG\_PENALTY}), lo que la descarta automáticamente de la población activa.

    \item \textbf{Selección por torneo:} Se emplea una estrategia de selección por torneo, donde se elige el mejor individuo entre grupos aleatorios de tamaño configurable. Este método favorece la diversidad y evita la convergencia prematura hacia soluciones subóptimas.

    \item \textbf{Operadores personalizados de cruce y mutación:} El cruce se realiza por segmentos aleatorios entre dos padres, y la mutación se aplica por gen con una probabilidad específica. Se evita asignar el mismo desarrollador en exceso o generar combinaciones inválidas.

    \item \textbf{Estrategia de elitismo parcial:} En cada generación, se conserva el mejor cromosoma hallado hasta ese momento, asegurando que el progreso evolutivo no se pierda. Sin embargo, solo se mantiene una élite mínima para no restringir la exploración del espacio de búsqueda.

    \item \textbf{Mecanismo de paciencia (estancamiento):} Si durante un número consecutivo de generaciones no se encuentra una mejor solución, el algoritmo se detiene anticipadamente, optimizando el tiempo de ejecución sin comprometer la calidad.
\end{itemize}

Este enfoque multiobjetivo permite una exploración equilibrada del espacio de soluciones, adaptándose a distintas prioridades definidas por el usuario y garantizando una asignación eficiente, válida y costo-efectiva de las tareas disponibles.

% \subsection{Algoritmo Genético Propuesto}

% El núcleo de optimización sigue un enfoque evolutivo, donde una población inicial de soluciones (cromosomas) se somete a iteraciones de evaluación, selección, cruce y mutación para mejorar la asignación de tareas.

% \begin{itemize}
%     \item Se parte de una población aleatoria de asignaciones.
%     \item Cada generación aplica operadores genéticos y se evalúan las soluciones.
%     \item La mejor solución se toma como sprint generado.
%     \item El proceso se repite hasta vaciar el backlog.
% \end{itemize}

% \noindent\textbf{Imagen:} \textit{(opcional)} flujograma simple del algoritmo. Ej: pseudodiagrama basado en 'while tareas\_pendientes:'.

\subsection{Representación de Soluciones}

En el contexto del algoritmo genético propuesto, cada solución candidata —también llamada cromosoma— representa una asignación completa de tareas a desarrolladores para un sprint específico. Esta asignación se codifica como una lista ordenada de identificadores de desarrolladores, donde la posición de cada elemento corresponde a una tarea según un orden topológico que respeta las dependencias del proyecto.

\textbf{Ejemplo de representación:}
Supóngase un conjunto de cuatro tareas ordenadas como $[T_0, T_1, T_2, T_3]$ y un cromosoma dado por la secuencia $[2, 1, 1, 3]$. Esto implica que:
\begin{itemize}
    \item La tarea $T_0$ se asigna al desarrollador con ID 2.
    \item La tarea $T_1$ se asigna al desarrollador con ID 1.
    \item La tarea $T_2$ también se asigna al desarrollador 1.
    \item La tarea $T_3$ se asigna al desarrollador con ID 3.
\end{itemize}

Dado que no todas las tareas pueden resolverse en un solo sprint —debido a restricciones de tiempo o dependencias—, el algoritmo aplica esta lógica de forma iterativa. Es decir, genera una secuencia de cromosomas (uno por sprint) hasta que se completa la asignación de todas las tareas del backlog. Cada uno de estos cromosomas constituye una solución parcial, y el conjunto completo representa la solución total del sistema.

Una forma simple de visualizar esta asignación es mediante una tabla que agrupe las tareas por sprint, detallando a qué desarrollador se asignó cada una de ellas y los tiempos estimados de ejecución.

\begin{center}
    \begin{tabular}{|c|c|c|c|c|}
        \hline
        \textbf{Sprint} & \textbf{Tarea} & \textbf{Desarrollador} & \textbf{Inicio (h)} & \textbf{Fin (h)} \\
        \hline
        1               & T0             & Dev 2                  & 0                   & 3                \\
        1               & T1             & Dev 1                  & 0                   & 5                \\
        1               & T2             & Dev 1                  & 5                   & 10               \\
        1               & T3             & Dev 3                  & 0                   & 4                \\
        \hline
        2               & T4             & Dev 2                  & 0                   & 4                \\
        2               & T5             & Dev 1                  & 0                   & 3                \\
        2               & T6             & Dev 3                  & 0                   & 2                \\
        \hline
    \end{tabular}
\end{center}

De esta forma, podemos ver claramente tanto la estructura de un cromosoma individual —con sus tareas asignadas, desarrolladores responsables y tiempos estimados— como el conjunto completo de cromosomas que, distribuidos por sprint, conforman una solución integral.

Este conjunto final de cromosomas constituye la salida principal del algoritmo, y será posteriormente interpretado por la interfaz del usuario, que se encargará de presentarlo de forma comprensible y útil para el usuario final.

% \subsection{Representación de Soluciones}

% Cada cromosoma representa una posible asignación de tareas a desarrolladores. Esta representación codifica:

% \begin{itemize}
%     \item Qué tareas se asignan a qué programadores.
%     \item El orden en que deben completarse (respetando dependencias).
%     \item El ajuste según las habilidades de cada desarrollador.
% \end{itemize}

% \noindent\textbf{Imagen:} \textit{(opcional)} ejemplo visual de un cromosoma o tabla de asignaciones (opcional si ya lo describes en texto).

\subsection{Evaluación de Soluciones (Función Objetivo)}

La calidad de cada solución candidata —representada por un cromosoma— es evaluada mediante una función objetivo compuesta que pondera múltiples criterios relevantes para la planificación ágil de proyectos. Esta función busca minimizar los siguientes aspectos:

\begin{itemize}
    \item \textbf{Makespan (50\%)}: tiempo total requerido para completar todas las tareas asignadas, considerando las dependencias y la disponibilidad de cada desarrollador.
    \item \textbf{Varianza de carga (25\%)}: medida de desequilibrio en la cantidad de trabajo distribuido entre desarrolladores. Cuanto menor sea, más balanceado es el plan.
    \item \textbf{Coincidencia de habilidades (20\%)}: penalización proporcional a la diferencia entre los niveles de habilidad requeridos por una tarea y los ofrecidos por el desarrollador asignado.
    \item \textbf{Costo total (5\%)}: estimado en función de las horas trabajadas por cada desarrollador y su tarifa horaria.
\end{itemize}

Todos estos valores son combinados mediante una fórmula ponderada de la siguiente forma:

\[
    \text{Fitness} = w_1 \cdot \text{Makespan} + w_2 \cdot \text{Varianza} + w_3 \cdot \text{SkillGap} + w_4 \cdot \text{Costo}
\]

Donde los pesos $(w_1, w_2, w_3, w_4)$ son definidos por el usuario a través de la interfaz del sistema y, por defecto, corresponden a $(0.5,\ 0.25,\ 0.2,\ 0.05)$.

Además, si una solución incumple restricciones críticas —como exceder la capacidad horaria de un desarrollador o asignar tareas con brechas de habilidad excesivas— se le asigna automáticamente una penalización severa definida como \texttt{BIG\_PENALTY}. Esto permite descartar de forma eficiente los cromosomas inviables sin necesidad de mantenerlos durante el proceso evolutivo.

\noindent\textbf{Pseudocódigo de evaluación:}

\begin{verbatim}
function evaluate_chromosome(chromosome):
    asignar tareas a desarrolladores según cromosoma
    calcular makespan total
    calcular varianza de carga entre desarrolladores
    calcular penalización por brechas de habilidades
    calcular costo total estimado

    if capacidad excedida o penalización muy alta:
        return BIG_PENALTY

    return w1*makespan + w2*varianza + w3*penalización + w4*costo
\end{verbatim}

Gracias a esta evaluación compuesta, el algoritmo puede priorizar soluciones más equilibradas y alineadas con los objetivos del sistema, al considerar simultáneamente múltiples criterios derivados de las restricciones del problema.

Esta función actúa como el eje central del proceso evolutivo, guiando la búsqueda hacia asignaciones óptimas dentro de un espacio de soluciones complejo y condicionado por las tareas y los perfiles de los desarrolladores disponibles.

% \subsection{Evaluación de Soluciones (Función Objetivo)}

% La función objetivo pondera múltiples criterios para determinar la calidad de una solución:

% \begin{itemize}
%     \item \textbf{Makespan (50\%)}: duración del sprint.
%     \item \textbf{Balance de carga (25\%)}: varianza en horas por developer.
%     \item \textbf{Coincidencia de habilidades (20\%)}: afinidad entre habilidades requeridas y ofrecidas.
%     \item \textbf{Costo (5\%)}: según tarifas por hora.
% \end{itemize}

% \noindent\textbf{Imagen:} \textit{(opcional)} fórmula matemática o pseudocódigo de evaluación.

\subsection{Operadores Genéticos Personalizados}

Los operadores han sido ajustados para respetar restricciones del problema:

\begin{itemize}
    \item \textbf{Selección:} por torneo.
    \item \textbf{Cruce:} combinación parcial de asignaciones entre padres.
    \item \textbf{Mutación:} reasignación de tareas con baja probabilidad.
    \item \textbf{Corrección:} validación para evitar duplicados y tareas inválidas.
\end{itemize}

\noindent\textbf{Imagen:} no requerida, puede ser solo texto.

\subsection{Interfaz y Configuración del Usuario}

El sistema permite que el usuario ajuste los parámetros del algoritmo genético desde una interfaz web amigable:

\begin{itemize}
    \item Población inicial
    \item Número de generaciones
    \item Tasa de mutación y cruce
    \item Tamaño del torneo
    \item Pesos para cada criterio de evaluación
\end{itemize}

\noindent\textbf{Imagen:} captura del frontend (panel de configuración).

\subsection{Ejemplo de Ejecución}

Se realizó una ejecución completa con el conjunto estático de tareas y desarrolladores. Se configuraron los parámetros del algoritmo y se generaron varios sprints:

\begin{itemize}
    \item Se obtuvo un conjunto de sprints con carga balanceada.
    \item Las tareas respetaron todas sus dependencias.
    \item El makespan fue razonable para el conjunto dado.
\end{itemize}

\noindent\textbf{Imagen:} \textit{opcional} captura de los resultados visualizados (por ejemplo, tabla con desarrolladores y tareas).


% La propuesta consiste en un sistema capaz de asignar automáticamente tareas a desarrolladores en base a un modelo evolutivo. El sistema considera el esfuerzo estimado por tarea, las dependencias entre ellas y la disponibilidad de los miembros del equipo.

% El algoritmo genético es ajustado para minimizar el desequilibrio de carga, respetar las restricciones impuestas por los sprints y optimizar el flujo de trabajo. La solución incluye un módulo de visualización que muestra la distribución sugerida.

% \vspace{0.5cm}

% \begin{tcolorbox}[colback=gray!10, colframe=black!30, title={Sugerencias para esta sección}]
%     \begin{itemize}
%         \item Explica brevemente qué hace tu solución y cómo se integra en un entorno ágil.
%         \item Muestra capturas o diagramas si ya has hecho avances de código.
%         \item Puedes separar en módulos: asignación, evaluación, visualización.
%         \item Justifica por qué tu propuesta mejora lo que ya existe.
%     \end{itemize}
% \end{tcolorbox}
\section{Conclusiones}

Este trabajo ha demostrado que es posible construir un sistema automatizado capaz de resolver la asignación de tareas en entornos ágiles SCRUM mediante un algoritmo genético multiobjetivo, integrando simultáneamente múltiples restricciones prácticas como habilidades requeridas, dependencias entre tareas, capacidades individuales y costos. La solución no solo logra generar asignaciones eficientes, sino que también respeta principios clave de metodologías ágiles como la iteración progresiva y la transparencia del proceso.

La implementación completa, compuesta por una interfaz gráfica en React, una API desarrollada en Flask y un motor de optimización en Python, valida que un enfoque modular y desacoplado es viable incluso en proyectos académicos, sentando las bases para futuras integraciones más complejas. Este diseño permitió realizar validaciones constantes a lo largo del desarrollo, ajustando y refinando componentes del algoritmo hasta alcanzar una versión estable y funcional.

Uno de los principales aprendizajes fue comprobar que, con una buena representación del problema y operadores genéticos bien diseñados, es posible obtener soluciones de alta calidad incluso en un dominio con restricciones duras y objetivos conflictivos. También se confirmó que la exposición clara de los parámetros de configuración al usuario (como tasas de cruce, mutación o pesos de la función objetivo) agrega valor al sistema, haciéndolo más flexible y comprensible.

Si bien existen limitaciones como el uso de datos estáticos para tareas y desarrolladores, estas no comprometen el valor de la propuesta como prueba de concepto sólida. El sistema desarrollado no solo cumple su objetivo académico, sino que representa un punto de partida válido para aplicaciones reales en entornos empresariales o educativos donde se requiere automatizar la gestión de tareas de forma inteligente.

% \section{Conclusiones}

% El enfoque basado en algoritmos genéticos ha demostrado ser una herramienta eficiente para la asignación de tareas en entornos SCRUM. Se ha logrado un balance adecuado en la distribución del trabajo, respetando restricciones y optimizando el tiempo total del proyecto.

% Este trabajo sienta las bases para aplicaciones reales en la industria del software, y abre la posibilidad de integrar técnicas similares en procesos más amplios de gestión ágil.

% \vspace{0.5cm}

% \begin{tcolorbox}[colback=gray!10, colframe=black!30, title={Sugerencias para esta sección}]
%     \begin{itemize}
%         \item Resume los aprendizajes clave del trabajo realizado.
%         \item Redacta al menos 3 conclusiones claras y bien separadas si es posible.
%         \item Puedes mencionar limitaciones y oportunidades de mejora.
%         \item Evita repetir el resumen o los objetivos textualmente.
%     \end{itemize}
% \end{tcolorbox}
\section{Recomendaciones}

A partir del análisis y la implementación realizada, se proponen las siguientes recomendaciones para futuras investigaciones o mejoras en sistemas SCRUM asistidos por algoritmos genéticos:

\begin{itemize}
    \item Integrar herramientas de visualización para facilitar la interpretación de las asignaciones realizadas por el algoritmo.
    \item Explorar técnicas híbridas que combinen algoritmos genéticos con heurísticas basadas en reglas de negocio.
    \item Validar el modelo con datos reales de proyectos ágiles en diferentes industrias.
    \item Evaluar el impacto del algoritmo en la satisfacción y productividad del equipo.
\end{itemize}

\vspace{0.5cm}

\begin{tcolorbox}[colback=gray!10, colframe=black!30, title={Sugerencia para esta sección}]
    Formula recomendaciones específicas, factibles y que se deriven directamente de los resultados de tu propuesta. Considera aspectos como mejoras técnicas, escalabilidad o nuevas direcciones de investigación.
\end{tcolorbox}

\section*{Bibliografía}
\begin{itemize}
  \item Goldberg, D. E. (1989). Genetic Algorithms in Search, Optimization and Machine Learning.
  \item Holland, J. H. (1992). Adaptation in Natural and Artificial Systems.
  \item DevOps Institute. (2021). DevOps Fundamentals.
\end{itemize}

\section*{Autores}
\begin{itemize}
  % \includegraphics[width=1.5cm]{imagenes/autor1.jpg}
  \item \textbf{Alumno ABC Soy}: Estudiante de ingeniería con interés en optimización y sistemas inteligentes.
        % \includegraphics[width=1.5cm]{imagenes/autor2.jpg}
  \item \textbf{Alumno DFC}: Apasionado por el análisis de datos y la automatización de procesos.
\end{itemize}


\end{document}


% \documentclass{article}
% \usepackage{styles/plantilla-profesor} % <-- Aquí se importa la plantilla

% \begin{document}

% \section*{Prueba de la plantilla del profesor}

% Este es un pequeño ejemplo para comprobar que todo está funcionando correctamente, incluso con la plantilla en el archivo styles. Aquí mostramos una fórmula matemática:

% \[
%   f(x) = x^2 + 3x + 4
% \]

% Y también un ejemplo de código con estilo personalizado:

% \begin{lstlisting}[style=matlabstyle, language=Python, caption={Algoritmo simple}]
% def cuadrado(x):
%     return x * x

% print(cuadrado(5))
% \end{lstlisting}

% \begin{tcolorbox}[contexto]
%   Este recuadro representa un contexto teórico que podrías usar en tu informe.
% \end{tcolorbox}

% \begin{tcolorbox}[ejercicio]
%   Desarrolla un experimento para verificar el rendimiento de tu algoritmo genético con distintos tamaños de población.
% \end{tcolorbox}

% \begin{tcolorbox}[conclusion]
%   La plantilla se ha compilado correctamente. Puedes comenzar a escribir tu informe a partir de esta base.
% \end{tcolorbox}

% \end{document}
