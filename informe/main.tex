\documentclass{article}

% === Plantilla del profesor ===
\usepackage[
left=4cm,
right=2.5cm,
top=2.5cm,
bottom=2.5cm
]{geometry}
\usepackage[utf8]{inputenc}
\usepackage[T1]{fontenc}
\usepackage{amsmath}
\usepackage{graphicx}
\usepackage{listings}
\usepackage{xcolor}
\usepackage{hyperref}
\usepackage{tcolorbox}
\usepackage{longtable}

% Colores para código
\definecolor{codegreen}{rgb}{0,0.6,0}
\definecolor{codegray}{rgb}{0.5,0.5,0.5}
\definecolor{codepurple}{rgb}{0.58,0,0.82}
\definecolor{backcolour}{rgb}{0.95,0.95,0.95}
\definecolor{codeblue}{rgb}{0,0,1}
\definecolor{codered}{rgb}{1,0,0}

% Estilo de código
\lstdefinestyle{matlabstyle}{
  backgroundcolor=\color{backcolour},   
  commentstyle=\color{codegreen},
  keywordstyle=\color{codeblue},
  stringstyle=\color{codered},
  basicstyle=\ttfamily\small,
  breakatwhitespace=false,         
  breaklines=true,                 
  captionpos=t,                    
  keepspaces=true,                 
  numbers=left,                    
  numbersep=5pt,                  
  showspaces=false,                
  showstringspaces=false,
  showtabs=false,                  
  tabsize=2,
  frame=single
}

% Cajas de contenido
\tcbset{
  conclusion/.style={
    colback=green!10,
    colframe=green!50!black,
    fonttitle=\bfseries,
    title=Conclusión
  },
  ejercicio/.style={
    colback=yellow!10,
    colframe=yellow!50!black,
    fonttitle=\bfseries,
    title=Ejercicio: Comportamiento de un Servidor Web Bajo Carga Constante
  },
  contexto/.style={
    colback=blue!10,
    colframe=blue!50!black,
    fonttitle=\bfseries,
    title=Contexto
  }
}

% === Inicio del contenido ===
\begin{document}

\section*{Prueba de la plantilla del profesor}

Este es un pequeño ejemplo para comprobar que todo está funcionando correctamente. Aquí mostramos una fórmula matemática:

\[
f(x) = x^2 + 3x + 2
\]

Y también un ejemplo de código con estilo personalizado:

\begin{lstlisting}[style=matlabstyle, language=Python, caption={Algoritmo simple}]
def cuadrado(x):
    return x * x

print(cuadrado(5))
\end{lstlisting}

\begin{tcolorbox}[contexto]
Este recuadro representa un contexto teórico que podrías usar en tu informe.
\end{tcolorbox}

\begin{tcolorbox}[ejercicio]
Desarrolla un experimento para verificar el rendimiento de tu algoritmo genético con distintos tamaños de población.
\end{tcolorbox}

\begin{tcolorbox}[conclusion]
La plantilla se ha compilado correctamente. Puedes comenzar a escribir tu informe a partir de esta base.
\end{tcolorbox}

\end{document}
